\chapter{Linux,Python和R语言的学习}

很多同学在学习生物信息学的开始往往顺着别人的Pipeline直接就做下来了,得到了结果但是对于流程控制的细节和原理知之甚少。 
“磨刀不误砍柴工”,我建议大家在上手一个项目之前至少弄清楚每个软件的基本原理,以便在调整参数的时候能够更加精细化,这往往能使你得到更精准的结论。 
但是也一定要注意的是,好的结论,光有好的分析过程是绝对不够的,好的数据质量往往决定了你的结论走向。
所以在进行分析之前也要与湿实验的合作者进行良好的沟通来了解实验的详情。



在读本书之前,我建议你先拥有良好的编程技能和数据分析技能,再进行一定的生物学学习,最后再进行分析。



关于编程,Linux的命令行是我们进行所有操作的基石,一定要学扎实,我在学习Linux的时候读过《鸟哥的Linux私房菜(基础篇)》这本书,鸟哥讲的很琐碎不过你要尽量学习过,至少过一遍,再去学习其他编程语言,这样你会更好地理解计算机和编程。 
可以租用服务器提供商的VPS搭建自己的个人博客这种方式来练习你的Linux基础技能,作者就是通过搭建网站的方式对Linux系统和命令行以及网络通讯有了一定的了解,这对接下来的学习非常重要。
当你拥有了一定的Linux使用经验之后,就会发现Bash命令行的局限性很多,语法也比较混乱,但是在简单的字符串处理上Bash脚本是效率非常高和方便易用的。当然你可能不知道什么叫Bash(或者什么是Shell),去百度或者Google自行学习。
在学习生物信息学的路上,最好的老师就是搜索引擎,希望你能牢牢记住这一点,尽量在求助别人之前先自己搜索一下,研究研究。 
我们谈到了Bash脚本的局限性,简单的字符串处理是它的强项,但是数据一旦开始复杂,Bash的使用就没有那么方便了,这时候我们就必须要掌握一到两门编程语言,作为一名即将进入生物信息学领域的研究人员,我强烈建议你熟练掌握Python和R两种编程语言,业内人员用的最多的就是Python和R,也有很多现成的程序,扩展包,算法实现和问题案例。 当然也可以学习Java,Julia和Go,但是会走不少弯路。 目前来说,Python的灵活和简单使得我们能够轻易实现想法,搭建分析框架(如Snakemake),得到计算结果,数据处理(Pandas)。 
而R是我们高效地进行统计学分析和数据可视化分析绕不开的好工具。 其实并不是离开R我们就不能进行数据分析,Python当然也可以实现很多,但是值得一提的是Bioconductor在生物信息学中占据了很重要的地位,而Bioconductor是R的扩展包,所以R是必须要掌握的基本技能。



总结一下,学习本书之前,我希望你的编程能力达到这种水平。


\begin{itemize}
	\item 
	第一,掌握Linux的基本使用,环境配置,达到能简单搭建个人博客水平,比如使用Markdown标记语言结合Nginx与Hexo。
	\item 第二,Python达到熟练进行数据分析的水平,掌握基本语法之后熟练运用os,pandas,numpy扩展包,对matplotlib合seaborn有一定的了解。
	\item 
	第三,R,掌握基本语法和向量、数组操作之后,尽量能够熟练操作frame,对ggplot2等扩展包有一定了解。 以上这些都是本书不会提起到的内容,但是默认你已经基本达到了上述水平,补充一句,不会就百度或Google,这真的很重要。
\end{itemize}


我会列出一些学习资源,希望你都能去学一学,多动手敲代码,多动脑思考,多尝试,如果你没有基础尽量按照下面的顺序去学,如果你有一定基础,选择性的去学:

\begin{itemize}
	\item
	《鸟哥的Linux私房菜》\footnote{http://linux.vbird.org/},重点学习,力求掌握Shell脚本编程的内容,了解前面几章计算机基础。
	\item
	《懂中文就会,黑马程序员Python基础视频教程》\footnote{https://www.bilibili.com/video/BV1ex411x7Em?from=search\&seid=5478123447193797430},内容设置非常好,建议全部学一遍,Pygame的部分可以不用学习。
	\item
%	《清华计算机博士带你学-Python金融量化分析》\footnote{https://www.bilibili.com/video/BV1i741147LS?from=search\&seid=3282011456454111951},Python中的扩展包Pandas,Numpy和Matplotlib可以按照这个课程去学,讲的比较容易懂。
	\item
	《R语言入门基础》\footnote{https://edu.csdn.net/course/detail/24913},这个讲的比较细致,有条理,R入门看这个就可以了。
	\item 
	《生信技能树-生信人应该这样学R语言》\footnote{https://www.bilibili.com/video/BV1cs411j75B?p=1},这个前面R基础讲的很乱,没有一定的R基础基本听不懂,但是结合了生物信息学特点去讲解,也有不少干货,后面还不错,所以建议先看上面的R语言入门基础,再来看这个课程,有耐心争取看完, 听不下去也无所谓。
\end{itemize}





